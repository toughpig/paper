% !TeX root = ../main.tex

\chapter{结论与展望}
% \section{总结}
随着RISC-V指令集架构的不断完善,RISC-V软件生态的不断蓬勃发展,未来的芯片设计与开发行业将越来越多地看到RISC-V的身影.本文依托实际的RISC-V芯片设计与开发项目,实现了RISC-V指令集模拟器,用于RISC-V架构软件的移植工作,以及针对具体微结构设计的验证工作完成了既定的功能需求,使得芯片开发团队可以脱离硬件平台进行系统软件移植/开发/测试工作,提供对真实硬件的功能模拟,以及丰富的调试手段,缩短软件开发周期.主要完成的工作如下:


(1)参照RISC-V用户手册,特权级架构手册,以及实际的硬件设计方案,对RISC-V标准拓展指令集共196条汇编指令进行C++功能函数模拟,其中大部分指令功能可以直接由RISC-V指令集手册获取,少部分指令(主要是特权架构指令)参照硬件设计团队具体的chisel代码进行翻译,对硬件设计的功能框架进行建模,确定模拟器的实现方案,规划具体功能模块的边界.


(2)实现RISC-V指令集模拟器前后端模块,添加丰富的调试手段,设计并实现平台级中断控制器PLIC和部分外设模拟,可以直接在模拟器进行外设驱动的开发,结合实际项目需求,添加UI可视化界面,进一步加强模拟器的易用性.在实际的芯片开发项目中帮助团队进行系统软件开发工作,并对处理器进行辅助验证.


本文开发的RISC-V指令集模拟器基本能够满足芯片开发团队的需求,但是后续的测试以及日常的使用过程中,发现本模拟器还是存在一些不足,需要在未来不断地进行改进和优化,主要包括以下两点:


(1)本模拟器没有提供保存快照的功能,有时调试过程中出现了异常的状态,因为随机的因素较难复现,所以保存快照的功能还是比较重要.


(2)本文设计的RISC-V指令集模拟器是依托具体的芯片开发项目,针对不同的微架构,需要进行微调才能够正常使用,可拓展性方面还有待提高,这方面的缺陷也是体系结构模拟器的通病,过分依托单一硬件.作为RISC-V架构的体系结构模拟器,在后续可以对功能模块进一步抽象,针对具体硬件相关的部分可以因地制宜具体实现,比如寄存器实现策略,中断控制器实现策略等.

% \section{未来工作}
