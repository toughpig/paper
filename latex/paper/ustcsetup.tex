% !TeX root = ./main.tex

\ustcsetup{
  title              = {RISC-V指令集模拟器的设计与实现},
  title*             = {Design and implementation of RISC-V instruction set simulator},
  author             = {王昊},
  author*            = {Wang Hao},
  speciality         = {软件工程},
  speciality*        = {Software Engineering},
  supervisor         = {汪增福~教授},
  supervisor*        = {Prof. Wang Zengfu},
  advisor            = {侯锐~研究员},
  advisor*           = {Prof. Hou Rui},
  professional-type  = {专业学位类型},
  professional-type* = {Professional degree type},
  % date               = {2017-05-01},  % 默认为今日
  % department         = {数学科学学院},  % 院系,本科生需要填写
  % student-id         = {PB11001000},  % 学号,本科生需要填写
  % secret-level       = {秘密},     % 绝密|机密|秘密|控阅,注释本行则公开
  % secret-level*      = {Secret},  % Top secret | Highly secret | Secret
  % secret-year        = {10},      % 保密/控阅期限
  %
  % 数学字体
  % math-style         = GB,  % 可选:GB, TeX, ISO
  math-font          = xits,  % 可选:stix, xits, libertinus
}

\usepackage{verbatim}
\usepackage{float}
\usepackage{listings}
\usepackage{xcolor}
\lstset{
  language=C++,  %代码语言使用的是matlab
  basicstyle=\small,
  % frame=shadowbox, %把代码用带有阴影的框圈起来
  % rulesepcolor=\color{red!20!green!20!blue!20},%代码块边框为淡青色
  % keywordstyle=\color{blue!90}\bfseries, %代码关键字的颜色为蓝色,粗体
  % commentstyle=\color{red!10!green!70}\textit,    % 设置代码注释的颜色
  showstringspaces=false,%不显示代码字符串中间的空格标记
  % numbers=left, % 显示行号
  % numberstyle=\tiny,    % 行号字体
  stringstyle=\ttfamily, % 代码字符串的特殊格式
  breaklines=true, %对过长的代码自动换行
  extendedchars=false,  %解决代码跨页时,章节标题,页眉等汉字不显示的问题
  % escapebegin=\begin{CJK*},escapeend=\end{CJK*},      % 代码中出现中文必须加上,否则报错
texcl=true}
  


% 加载宏包

\usepackage{makecell}

\usepackage{array}


% 定理类环境宏包
\usepackage{amsthm}

% 插图
\usepackage{graphicx}

% 三线表
\usepackage{booktabs}

% 跨页表格
\usepackage{longtable}

% 算法
\usepackage[ruled,linesnumbered]{algorithm2e}

% SI 量和单位
\usepackage{siunitx}

% 参考文献使用 BibTeX + natbib 宏包
% 顺序编码制
\usepackage[sort]{natbib}
\bibliographystyle{ustcthesis-numerical}

% 著者-出版年制
% \usepackage{natbib}
% \bibliographystyle{ustcthesis-authoryear}

% 本科生参考文献的著录格式
% \usepackage[sort]{natbib}
% \bibliographystyle{ustcthesis-bachelor}

% 参考文献使用 BibLaTeX 宏包
% \usepackage[style=ustcthesis-numeric]{biblatex}
% \usepackage[bibstyle=ustcthesis-numeric,citestyle=ustcthesis-inline]{biblatex}
% \usepackage[style=ustcthesis-authoryear]{biblatex}
% \usepackage[style=ustcthesis-bachelor]{biblatex}
% 声明 BibLaTeX 的数据库
% \addbibresource{bib/ustc.bib}

% 配置图片的默认目录
\graphicspath{{figures/}}

% 数学命令
\makeatletter
\newcommand\dif{%  % 微分符号
  \mathop{}\!%
  \ifustc@math@style@TeX
    d%
  \else
    \mathrm{d}%
  \fi
}
\makeatother
\newcommand\eu{{\symup{e}}}
\newcommand\iu{{\symup{i}}}

% 用于写文档的命令
\DeclareRobustCommand\cs[1]{\texttt{\char`\\#1}}
\DeclareRobustCommand\pkg{\textsf}
\DeclareRobustCommand\file{\nolinkurl}

% hyperref 宏包在最后调用
\usepackage{hyperref}
